\subsection{Motivation}
This project, RISC-V: A Didactic Platform, embarks on the pursuit of simplifying complex computer architecture concepts, 
targeting an academic audience who is striving to comprehend the intricate world of processor design. 
It specifically focuses on the creation of a RV32I processor, based on the RISC-V instruction set architecture.

RISC-V, an open standard instruction set architecture (ISA), has increasingly gained popularity due to its simplified design and 
flexibility for customization. While its usage has been witnessed across a multitude of applications, its potential as an educational tool remains largely unexplored. 
The objective of this project is to bridge this gap by providing a detailed, comprehensible, and implementable design of an RV32IM processor, leveraging the RISC-V architecture.

This project not only contributes to the existing body of knowledge around RISC-V and processor design but also aims to democratize access to information on complex computing architectures. 
Through this platform, users will not only learn the theory behind processor design but will also gain valuable hands-on experience with the actual implementation, 
enabling them to better understand the interplay between hardware and software in a computing system.

In an era where the understanding of computer architecture is pivotal for both hardware and software development, 
a project like RISC-V: A Didactic Platform can serve as a crucial resource. 
By harnessing the power of the RISC-V architecture and presenting it through a user-friendly and accessible platform, 
we hope to elevate the understanding of processor design to new heights.