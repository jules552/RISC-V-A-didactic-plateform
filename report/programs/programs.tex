There are two main ways of developing programs for my CPU that I've set up.\\

The first one is using assembly code or pseudo assembly code and using this website\cite{riscv_online_assembler} and it will output 
the result in hexadecimal that is readable by the ROM, and the same for the python script\cite{riscv_python_assembler} but it uses a 
different syntax that is very close to actual RISC-V assembly.\\

The second one is using C code inside the \texttt{c\_code} folder of the project. But this solution requires a bit more work before 
using it since you have to compile the RISC-V toolchain with the correct architecture, so here RV32IM, then correctly give the variables 
to your bash environment and then you can compile the C code using the Makefile inside the \texttt{c\_code} folder. Here a detailed guide 
that combines the document from the RISC-V toolchain\cite{riscv_gnu_toolchain} and the tutorial I've found about bare metal RISC-V development
\cite{riscv_gnu_toolchain_baremetal_tutorials}.

\begin{enumerate}[label={\textbullet}]
    \item Clone the toolchain from \href{https://github.com/riscv-collab/riscv-gnu-toolchain}{here}.
    
    \begin{verbatim}
    git clone https://github.com/riscv/riscv-gnu-toolchain.git
    \end{verbatim}
    
    \item Navigate into the cloned directory.
    
    \begin{verbatim}
    cd riscv-gnu-toolchain
    \end{verbatim}
    
    \item Run the configuration script.

    \begin{verbatim}
    ./configure --prefix=/opt/riscv --with-arch=rv32im
    \end{verbatim}
    
    \item Compile the toolchain. This step may take a while.
    
    \begin{verbatim}
    make
    \end{verbatim}
    
    \item Once the compilation is done, add the toolchain to your PATH in your bash configuration file.

    \begin{verbatim}
    echo 'export PATH=$PATH:/opt/riscv/bin' >> ~/.bashrc
    source ~/.bashrc
    \end{verbatim}
\end{enumerate}

After that, you should be able to use the toolchain. To test it, you can try making a simple program like the ones in the 
$c\_code$ folder of the project. You can simply compile them with the following command:

\begin{verbatim}
    make
\end{verbatim}


The only thing that was needed after that to execute the C code is creating a linker file that indicates to the compiler where 
to put the different sections of the code and it has been done by me to facilitate the process and was quite simple to do actually.
The only "issue" you'll have is that since the architecture RV32IM is so small, the C standard library is not included in the toolchain
and you'll have to implement it yourself if you want to use it but I still think it is less painful than using assembly code.\\