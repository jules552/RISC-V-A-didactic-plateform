I've also put as disposition an "easy" way to generate programs to use as input to the ROM of the processor. You still have to install 
the RISC-V toolchain to use it and configure correctly the bash or zsh file to indicate where the toolchain is installed.
Installing the toolchain can take quite a while since you'll have to compile it and can take more than 45min and will highly depend
on what CPU you have. \\

Here is a step-by-step guide to installing the toolchain:

\begin{enumerate}[label={\textbullet}]
    \item Clone the toolchain from \href{https://github.com/riscv-collab/riscv-gnu-toolchain}{here}.
    
    \begin{verbatim}
    git clone https://github.com/riscv/riscv-gnu-toolchain.git
    \end{verbatim}
    
    \item Navigate into the cloned directory.
    
    \begin{verbatim}
    cd riscv-gnu-toolchain
    \end{verbatim}
    
    \item Run the configuration script.

    \begin{verbatim}
    ./configure --prefix=/opt/riscv --with-arch=rv32im
    \end{verbatim}
    
    \item Compile the toolchain. This step may take a while.
    
    \begin{verbatim}
    make
    \end{verbatim}
    
    \item Once the compilation is done, add the toolchain to your PATH in your bash or zsh configuration file.

    \begin{verbatim}
    echo 'export PATH=$PATH:/opt/riscv/bin' >> ~/.bashrc
    source ~/.bashrc
    \end{verbatim}
\end{enumerate}

After that, you should be able to use the toolchain. To test it, you can try making a simple program like the ones in the 
$c\_code$ folder of the project. You can simply compile them with the following command:

\begin{verbatim}
    make
\end{verbatim}

That will generate a .hex file that you can use as input to the ROM of the processor. It will also give you the assembly
code of the program in the .s file so that you can see what the compiler generated. \\